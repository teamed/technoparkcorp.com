%%
%% Copyright TechnoPark Corp., 2010
%% @version $Id$
%%

<?=$this->estimate->estimator?> used \textbf{function-point analysis}
method, which is based on the assumption that 
software code consists of atomic elements, called function points. Each
feature has a certain amount of function points ($FP_i$), with some complexity ($C_i$). 
We multiply the amount of function points to complexities and summarize the result. What we get is
called unadjusted function points ($UFP$ = \texttt{<?=$this->estimate->ufp?>}).
Roughly, in object-oriented programming function points represent
class methods. Using these numbers calculate a forecasted number of thousands of software lines of code
($KSLoC$ = \texttt{<?=$this->estimate->ksloc?>}).
Project size in people-months ($PM$) is calculated with a formula from 
\href{http://en.wikipedia.org/wiki/COCOMO}{COCOMO-II} ($A$ and $B$
are found experimentally):

\begin{equation}\begin{gathered}
PM = A \times KSLoC^{\displaystyle(B + 0.01 \times \sum_{j=1}^5 SF_j)}
    \times {\displaystyle \prod_{i=1}^n EM_i} \\
PM = <?=sprintf('%0.2f', $this->estimate->aFactor)?>
    \times <?=$this->estimate->ksloc?>^{<?=sprintf('%0.2f', $this->estimate->bFactor)?>} 
    = <?=sprintf('%0.2f', $this->estimate->pm)?> 
\end{gathered}\end{equation}

The result
(we multiplied $PM$ to <?=$this->estimate->daysInMonth?> --- the amount of staff-hours
in one people-month) is \textbf{<?=$this->estimate->hours?> staff-hours}.

