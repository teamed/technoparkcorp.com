%%
%% Copyright TechnoPark Corp., 2010
%% @version $Id$
%%

\section*{Rough-Order-of-Magnitude (ROM) Estimate}

We use a number of industry-wide estimating methods,
and solicit numbers from different estimators. Each method
produces its own estimates. The arithmetic average of those estimates 
is the final result. On the next <?=$this->sheet->total?>{} pages
you will find information received from estimators, with our
comments and calculations.

The bottom line is:

\begin{tabular}{>{\raggedright}p{25em}r}
    Estimator & Hours \\
    \hline
    <? foreach ($this->sheet->estimates as $estimate): ?>
        <?=$this->tex($estimate->estimator)?>{}
        <?=$this->tex($estimate->promo)?>{} &
        <?=$estimate->hours?>{} \\
    <? endforeach; ?>
    \hline
\end{tabular}

Math average of the numbers received is <?=$this->sheet->hours?>,
and the ROM interval is
<?=$this->sheet->lowBoundary?>--<?=$this->sheet->highBoundary?>{}
staff-hours.

Read more information about <?=$this->href('process/cost/rom', 'ROM Estimate')?>{}
and our <?=$this->href('process/cost', 'two-step quotation process')?>.

<? foreach ($this->sheet->estimates as $estimate): ?>
    \clearpage
    \subsection*{Estimate by ``<?=$this->tex($estimate->estimator)?>''}
    
    {\setstretch{1.2}\begin{verbatim}
<?=wordwrap(implode("\n", $estimate->lines), 70)?>
    \end{verbatim}}
    
    <?=$this->render('ROM/' . $estimate->proposalFile)?>
<? endforeach; ?>
